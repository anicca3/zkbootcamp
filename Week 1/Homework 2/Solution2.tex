\documentclass{article}
\usepackage[utf8]{inputenc}
\usepackage[english]{babel}
\usepackage[]{amsthm} %lets us use \begin{proof}
\usepackage[]{amssymb} %gives us the character \varnothing
\usepackage{amsmath}
\usepackage{hyperref}

\topmargin=-1.2in
\evensidemargin=0in
\oddsidemargin=0in
\textwidth=6.4in
\textheight=9.2in

\title{ZK Bootcamp: Homework 2}
\author{Kuriakin Zeng}
\date\today
%This information doesn't actually show up on your document unless you use the maketitle command below

\begin{document}
\maketitle %This command prints the title based on information entered above

%\section*{Math Introduction}

\subsection*{Problem 1}
\begin{proof}
$B = \{0,1\}$ and the operation $\oplus$ is a group
\begin{itemize}
\item Closure. It's easy to see that from the rules that the output is $\in B$
\item Associativity. $0 \oplus 1 = 1 \oplus 0 = 1$
\item Identity. The identity element is 0 since for $b \in B$, $b + 0 = b$ and it's unique
\item Inverse element. $0^{-1} = 0$ and $1^{-1} = 1$
\end{itemize}
\end{proof}

\subsection*{Problem 2}
\begin{itemize}
\item Odd squares are $\equiv 1 \pmod 8$
\begin{proof}
Squares are produced by adding consecutive odd numbers. An odd square is produced by summing the odd number of odd numbers: the first odd square is 1, the second odd square is $1+3+5 = 9$, and the subsequent odd squares is $1+3+5+(3+4n)+(5+4n)$ where $n = \{1, 2, 3, ...\} $
It's easy to see that the first two odd squares $\equiv 1 \pmod 8$ and $1+3+5+(3+4n)+(5+4n)$ can be rewritten as $1+3+5+(8+8n)$. Since $8+8n \equiv 0 \pmod 8$, it must be that all odd squares $\equiv 1 \pmod 8$
\end{proof}
\item Even squares are $\equiv 0 \pmod 8$
\begin{proof}
An even square is produced by summing the even number of odd numbers: the first even square is $1+3=4$, and the subsequent is $1+3+(1+4n)+(3+4n)$ where $n = \{1,2,3,...\}$
We can rewrite the equation as $8+8n$, thus it must be that even squares are $\equiv 0 \pmod 8$
\end{proof}
\end{itemize}

\subsection*{Problem 3}
We generate a random private key (n) and use a point on elliptic curve (G) [secp256k1 $\rightarrow y^2 = x^3 + 7$] to generate the public key $n \cdot G$. The hash of the public key is the Bitcoin address. Helpful resource: \href{https://www.youtube.com/watch?v=muIv8I6v1aE}{https://www.youtube.com/watch?v=muIv8I6v1aE}

\subsection*{Problem 4}
The three represent the worst case complexity of a function, i.e. the amount of resources required to run a function in the worst case scenario
\begin{itemize}
\item $O(n)$ means the complexity grows linearly with respect to the input size
\item $O(1)$ means the complexity doesn't grow with respect to the input size
\item $O(log n)$ means the complexity grows logarithmically with respect to the input size 
\end{itemize}

\subsection*{Problem 5}
The best case for proof size is $O(1)$ 

\clearpage %Gives us a page break before the next section. Optional.

\end{document}